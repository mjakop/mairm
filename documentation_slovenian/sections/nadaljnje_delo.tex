Projekt je objavljen na google code (http://code.google.con/p/mairm/) in kot tak je na voljo vsem, ki bi radi kaj dodali/spremenili. Torej odprte so vse opcije za sodelovanje drugih ljudi in tem uresni�evanje njihovih idej. Kot uporabnika in razvijalca sistema sva pri�la do ugotovitve, da bi bilo dobro za bolj�o uporabni�ko izku�njo narediti �e slede�e stvari:
\begin{enumerate}
\item V XML datoteko uvesti pogoje glede na trenutno aktivno aplikacijo. Tako bi se lahko zagotovilo, da bi ista kretnja imela razli�en pomen v razli�nih programih. Npr. kretnja v desno bi v predvajalniku glasbe prestavila na drugo pesem, medtem ko v pregledovalniku slik na naslednjo sliko.
\item Nekako poskrbeti, da bi sistem sam izbolj�eval svoje znanje o kretnjah (brez sodelovanja uporabnika pri temu).
\item Prestaviti delovanje programa na vi�ji prioritetni nivo (ker se druga�e zna zgoditi, da ob veliki zasedenosti ra�unalnika celoten sistem deluje upo�asnjeno in ne tako kot prava mi�ka), �eprav je ta pojav redek.
\end{enumerate}
Bistveno za nadaljnje delo je to, da morava poskrbeti, da bo ve� ljudi spoznalo najin projekt in ga morebiti za�elo uporabljati. Le-tako bo smiselno dodajati nove funkcionalnosti in izbolj�ave. �e neka re�itev nima uporabnikov, vsak (pa naj bo �e tako dober) projekt slej ko prej umre. Kar je v�asih �koda.
