Cilj seminarske naloge je izdelati delujo� sistem in se pri tem nau�iti nove stvari iz podro�ij, ki jih seminarska naloga obsega. Bolj konkretni cilji sistema so:
\begin{itemize}
\item omogo�iti enostavno uporabo sistema,
\item s pomo�jo premikov telefona upravljamo z ra�unalni�ko mi�ko (premiki levo, desno, gor, dol),
\item omogo�iti uporabniku drsenje (scrolling) preko strani s preprostim nagibanjem telefona naprej oziroma nazaj,
\item omogo�iti uporabniku vnos kraj�ih besedil(za dalj�a besedila telefonska tipkovnica ni preve� primerna),
\item omogo�iti uporabniku, da ra�unalnik "nau�i" kako naj bi neka kretnja izgledala,
\item tako "nau�en" ra�unalnik naj bi potem ob uporabnikovi izvedbi kretnje naredil dolo�eno akcijo (npr. zagnal dolo�en program) in
\item vse skupaj narediti z odprtokodnimi orodji/knji�njicami ipd.
\end{itemize}
Ob vseh zgoraj podanih ciljih se pojavi vpra�anja kak�nim zahtevam je potrebno zadostiti, da bo uporaba kon�ne re�itve sploh mo�na. Zahteve za delovanje sistema naj bi bile slede�e:
\begin{itemize}
\item spodoben ra�unalnik z urejeno Java podporo,
\item bluetooth vmesnik,
\item mobilni telefon z merilnikom pospe�ka: v najinem primeru sva se omejila na telefona znamke Nokia s Symbian operacijskim sistemom, kljub temu je prenosljivost na ostale mobilne platforme trivialna.
\end{itemize}
Kot je opaziti iz zahtev re�itev deluje na vseh operacijskih sistemih, kjer je podpora za programski jezik Java. Samo delovanje sistema je bilo testirano na Windows in Linux platformi, kjer je deloval brez te�av.\newline\newline
Po preu�itvi podobnih re�itev na spletu sva pri�la do ugotovitve, da re�itve, ki bi vklju�evala vse najine cilje �e ni na voljo. Obstoje�e re�itve omogo�ajo ve� ali manj samo katero izmed funkcionalnosti/ciljev, ki sva si jih midva zadala. Ena izmed tak�nih re�itev je WiiGee. WiiGee omogo�a u�enje in razpoznavo kretenj. Na �alost/sre�o je ta re�itev osnovana za lastnike igralne konzole Wii. 