Sam protol je v JSON formatu in te�e preko Bluetooth Serial Port ter je zelo enostaven. V trenutni verziji je podprta samo enosmerna komunikacija, saj potrebe po dvosmerni komunikaciji (zaenkrat) ni bilo.
\subsubsection{Uporabljeni podatkovni tipi}
boolean  = true ali false\newline\newline
double = decimalno �tevilo\newline\newline
buttonState = up ali down\newline\newline
buttonStateMiddle = up ali down ali scrolling\newline
scrolling pomeni, da telefon preide iz na�ina nadziranja mi�ke na na�in nadziranja scrollerja preko nagibanja\newline\newline
keys = ASCII ali ENTER ali BACKSPACE ali ...\newline
Dodatne tipke so stvar dogovora in potrebe\newline\newline
modifiers = SHIFT in/ali ALT in/ali CONTROL

\subsubsection{Format za GESTURE mode}
\textbf{Oblika:}
\begin{lstlisting}[caption=Oblika formata za GESTURE mode]
{"gesture":{"x":"double","y":"double","z":" double ","start":"boolean","end":"boolean"}}
\end{lstlisting}
\textbf{Razlaga:}\newline
x = vrednost akselometra po x-osi. \newline
Privzeta vrednost: 0.0\newline\newline
y = vrednost akselometra po y-osi.\newline
Privzeta vrednost: 0.0\newline\newline
z = vrednost akselometra po z-osi.\newline
Privzeta vrednost: 0.0\newline\newline
start = dolo�a ali prejeti podatki pomenijo za�etek gesture ali ne. Vrednost true ima lahko samo pri prvem sporo�ilu nekega gesture.\newline
Privzeta vrednost: false\newline\newline\newline
end = dolo�a ali prejeti podatki pomenijo konec gesture ali ne. Vrednost true ima lahko le pri zadnjem sporo�ilu za nek gesture. \newline
Privzeta vrednost: false


\subsubsection{Format za MOUSE mode}
\textbf{Oblika:}
\begin{lstlisting}[caption=Oblika formata za MOUSE mode]
{"mouse":{"x":"double","y":"double","z":"double","leftbutton":"buttonState","middlebutton":"buttonStateMiddle","rightbutton":"buttonState"}}
\end{lstlisting}
\textbf{Razlaga:}\newline
x = vrednost akselometra po x-osi. \newline
Privzeta vrednost: 0.0\newline\newline
y = vrednost akselometra po y-osi.\newline
Privzeta vrednost: 0.0\newline\newline
z = vrednost akselometra po z-osi.\newline
Privzeta vrednost: 0.0\newline\newline
leftbutton = levi mi�kin gumb\newline
Privzeta vrednost: up\newline\newline
middlebutton = srednji mi�kin gumb\newline
Privzeta vrednost: up\newline\newline
rightbutton = desni mi�kin gumb\newline
Privzeta vrednost: up\newline

\subsubsection{Format za KEYBOARD mode}
\textbf{Oblika:}
\begin{lstlisting}[caption=Oblika formata za KEYBOARD mode]
{"keyboard": {"key":"keys","modifiers":[modifiers]}} 
\end{lstlisting}
\textbf{Razlaga:}\newline
key = tipka, ki naj bi bila pritisnjena na ra�unalni�ki tipkovnici. Posebni znaki morajo biti zapisani s 
polnim imenom (npr. "). 
modifiers = v array se navede katere tipke so dodatno �e pritisnjene. Npr. za veliki A moramo dodati v 
modifiers element "SHIFT". 

