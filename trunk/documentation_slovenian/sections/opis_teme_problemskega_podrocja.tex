Tema seminarske naloge je izdelava virtualne naprave vhodno/izhodnega sistema, ki omogo�a naslednje stvari:
\begin{itemize}
\item Upravljanje mi�ke preko nagibanja telefona in pritiskanja tipk
\item Vna�anje kratkih besedil preko tipkovnice mobilnega telefona
\item Zaznavanje kretenj v 3D prostoru (nekaj nari�emo s telefonom v zraku, sistem prepozna kaj to je in izvede ustrezno akcijo)
\end{itemize}

�e pogledamo malo bolj v podrobnosti in ne samo te to�ke ugotovimo, da izvedba celote ni tako trivialna kot se sprva zdi. Problemati�no je predvsem prepoznavanje kretenj v 3D prostoru.